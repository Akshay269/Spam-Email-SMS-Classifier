\cleardoublepage
\section{LIMITATIONS AND FUTURE ENHANCEMENTS}
% spandan
\subsection{Limitations}
This project is highly planned and acted upon from the beginning. Nevertheless, the project had to face some of the limitations due to various factors. Different aspects of the projects such as nature of data, visualisation methods, data storage method and so on have their own limitations. Some of the limitations faced by the project are:-

\begin{enumerate}
  \setlength\itemsep{1.5em}
	\item The sensors used for the purpose of taking weather and pollution data are highly expensive. Sometimes, the money spent in the sensors don't even justify the accuracy and precision of data given by the sensors. Also, increasing number of sensors covers the entire budget of the project.
	\item Although the sensors used in the project have been re-calibrated for the purpose of data collection in the project, data still contained a number of errors and missing values. 
	\item The attempts have been made to make the visualisation effective for the users with the help of map based visualisation along with charts and table. Still, different users confronted with the same data visualisations may not necessarily draw the same conclusion, depending on their previous experiences and particular level of expertise.
	\item Although the use of big data architecture has made the access of large amount of spatio-temporal data easier and faster, lack of resources for extension of nodes in the architecture has limited us to efficiently utilise the power of big data technology.
	\item For creation of model, the use of powerful processing resources is required. They are expensive and are not easily available.
	\item The installation of sensors in the vehicle has hampered the accuracy of data as some of the vehicle parameters like heat produced by the engine, wind due to high speed and so on affect the working of the sensor.
	\item The data logging technology used in the sensors cause difficulty in examining the working status of the sensors.
\end{enumerate}

\newpage
\subsection{Future Enhancements}

It is the nature of projects in the field of computer science and information technology to require changes and modifications as demand changes and technology advances. This project will be enhanced in the future for better visualisation of spatio-temporal weather and pollution data.

\begin{enumerate}
  \setlength\itemsep{1.5em}
    \item Self-powered sensors that are powered using the perpetual natural resources like solar energy shall be used.

	\item Quality sensors with better accuracy and precision, as well as ability to resist heat and harsh environmental conditions will be used in the future. The use of such sensor will measure the data with low missing values, errors and with greater stability.
	
	\item Existing database i.e. PostgreSQL used for data storage can be replaced by big data architecture to be able to meet the requirement of operations of huge amount of spatio-temporal data.
	
	
	\item The system application is limited to web browser. The system made for the visualisation of data shall be extended to the mobile application as well.
	
	\item The near-real time Pi-sensor data shall be used for modelling and finding the pattern.
	

	\item  Sensors can be made to communicate with each other through the IOT technology. This can save external network bandwidth as well as give rise to other interesting applications.
	
		\item  Distributed Computing architecture can be made to utilise the processing power of sensor devices when idle for on-device data processing and anomaly detection.
	          
\end{enumerate}